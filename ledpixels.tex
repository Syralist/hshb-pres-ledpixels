\documentclass[aspectratio=1610,svgnames]{beamer}

\usepackage{lmodern}
\usepackage[T1]{fontenc}
\usepackage[ngerman]{babel}
\usepackage{selinput}
\SelectInputMappings{%
   adieresis={ä},
   germandbls={ß}
   }

\usetheme{PaloAlto}  %% Themenwahl

\setbeamercovered{transparent}
%\setbeamertemplate{footline}[frame number]
\usecolortheme{spruce}		% grün
\usecolortheme[named=MSUgreen]{structure}
\newcommand{\hshblogo}{\includegraphics[width=1.3cm]{space-logo}}
\newcommand{\divider}[1]{\begin{frame} %
\begin{alertblock}{} %
\centering\usebeamerfont{section title}#1 %
\end{alertblock} %
\end{frame}}
 
\title{Ein LED Videogame Display}
\author{Thomas Helmke}
\date{02.09.2015}
\logo{\includegraphics[width=1.1cm]{space-logo}}
 
\begin{document}
\maketitle
\frame{\tableofcontents}

\section{Einleitung}
\divider{\insertsection}
\begin{frame}[<+->] %%Eine Folie
	\frametitle{Worum geht es?} %%Folientitel
	\begin{itemize}
		\item Ein selbstgebautes Videospiel
		\item Gemeinschaftsprojekt des Vereins
		\item Open Source bei Hard- und Software
	\end{itemize}
\end{frame}

\section{Hardware}
\divider{\insertsection}
\begin{frame}[<+->]
    \frametitle{Hardware}
    \begin{itemize}
        \item Display aus 1800 WS2812b RGB LEDs
        \item Rahmen selbst konstruiert und gebaut
        \item Gefrästes Lochraster, darüber Diffusorpapier
        \item Leistungsaufnahme maximal 350 Watt
    \end{itemize}
\end{frame} 
\begin{frame}[<+->]
    \frametitle{Display Ansteuerung}
    \begin{itemize}
        \item Urprüngliche Version mit Arduino Due
        \item Alle LEDs an einer Datenleitung
        \item Aktuelle Version mit Teensy 3.1
        \item höhere Framerate durch DMA
    \end{itemize}
\end{frame} 
\begin{frame}[<+->]
    \frametitle{Controller Box}
    \begin{itemize}
        \item klassischen Arcadecontrollern nachempfunden
        \item 2 Digitale Joysticks
        \item 6 Player Buttons plus 2 Extra Buttons
        \item Kasten selbst konstruiert und gebaut
    \end{itemize}
\end{frame} 
\begin{frame}[<+->]
    \frametitle{Controller Box}
    \begin{itemize}
        \item Inputs an einem Arduino Due
        \item Ursprüngliche Version mit seriellem Kabel
        \item aktuell Verbindung über Seriell auf Bluetooth Adapter
        \item Betrieb mit recycleten Akkus
    \end{itemize}
\end{frame} 

\section{Software}
\divider{\insertsection}
\begin{frame}[<+->]
    \frametitle{Arduino Version}
    \begin{itemize}
        \item Fork des Spiels von Kris Temmerman
        \item Spiellogik und Displayansteuerung auf einem Arduino Due
        \item Minimale Anpassung nötig für unseren Nachbau
        \item keine Weiterentwicklung weil Arduino Programmierung nicht jedermanns Sache
    \end{itemize}
\end{frame}
\begin{frame}[<+->]
    \frametitle{Teensy Version}
    \begin{itemize}
        \item Teensy steuert Display über DMA
        \item Zwei Datenleitungen für je eine Hälfte
        \item Empfängt RGB Information über serielle Schnittstelle
        \item Spieleprogrammierung mit beliebiger Sprache möglich
    \end{itemize}
\end{frame}
\begin{frame}[<+->]
    \frametitle{PyGame FTW}
    \begin{itemize}
        \item aktuelles Framework basiert auf PyGame
        \item erstes Displaymodul direkt aus Python auf die Serielle Schnittstelle
        \item neuste Version: Displayserver programmiert mit Node.js
        \item läuft alles auf einem Raspberry Pi
    \end{itemize}
\end{frame}
\begin{frame}[<+->]
    \frametitle{Unser Framework}
    \begin{itemize}
        \item basiert auf PyGame
        \item stellt ein Display Modul für den Displayserver bereit
        \item Controllserver empfängt Bluetooth Daten und simuliert einen Joystick
        \item Event Wrapper zur einfachen Verarbeitung der Signale vorhanden
    \end{itemize}
\end{frame}

\section{Weitere Infos}
\divider{\insertsection}
\begin{frame}
    \frametitle{Weitere Infos}
    \begin{itemize}
        \item \url{https://github.com/Syralist/hshb-pres-ledpixels}
        \item \url{https://wiki.hackerspace-bremen.de/projekte/videogame/start}
        \item \url{https://github.com/HackerspaceBremen}
        \item \url{https://gist.github.com/jh0ker/8a63a66d368d7b48c89d}
    \end{itemize}
\end{frame}
\end{document}
